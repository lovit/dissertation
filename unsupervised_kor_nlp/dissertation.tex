% Set parameters
% set font size
\documentclass[11pt]{article}

% set line height
\renewcommand{\baselinestretch}{2}

% line number
\usepackage{lineno}

% today
\renewcommand{\today}{}

% set margin
\usepackage{geometry}
\geometry{a4paper, left=15mm, right=15mm, top=20mm, bottom=20mm}

% for using Korean
\usepackage{kotex}

% for using multi-languages
\usepackage[english]{babel}

% for urls
\usepackage{hyperref}

% for equation
\usepackage{amsmath}

% for fixing figure position
\usepackage{float}

% for citation
\usepackage{natbib}
\bibliographystyle{apa}
\setcitestyle{authoryear,open={(},close={)},citesep={;}}

% for drawing table
\usepackage{array,multirow,graphicx,rotating,booktabs}
\usepackage[table,xcdraw]{xcolor}

% for tabularx
\usepackage{tabularx}
\newcolumntype{b}{X}
\newcolumntype{s}{>{\hsize=.5\hsize}X}
\renewcommand\arraystretch{0.8} \setlength\minrowclearance{0.8pt}

%%%%%%%%%%%%%%%%%%%%%%%%%%%%%%%%%%%%%%%%%%%%%%%%%%%%%%%%%%%%%%%%%%%%%%%%%%%%%%%
\begin{document}

\title{Unsupervised Korean Natural Language Processing to Solve Out-of-Vocabulary and Dearth of Data}
\author{Hyunjoong Kim}

\maketitle
\smallskip

%%%%%%%%%%%%%%%%%%%%%%%%%%%%%%%%%%%%%%%%%%%%%%%%%%%%%%%%%%%%%%%%%%%%%%%%%%%%%%%
\section{Introduction}

자연어처리 (natural language processing) 는 사람의 언어를 컴퓨터가 이용할 수 있는 형태의 정보로 변환하고, 이를 이용하여 과업들 (tasks) 을 수행하는 분야이다.
자연어처리에서 수행하는 과업은 다양하다.
품사 판별과 형태소 분석은 텍스트를 단어열로 분해하는 전처리 과업에 해당한다.
단어의 특정 품사나 의미를 이해하는 객체명 인식과, 키워드나 핵심 문장을 이용하여 문서 집합을 요약하는 정보 추출 과업도 자연어처리에 포함된다.
또한 사용자의 질문에 대해 적절한 답변을 탐색하는 질의 응답도 자연어처리 과업에 포함된다.

자연어처리 과업의 많은 부분은 머신 러닝 기법을 이용한다.
머신 러닝은 학습 방식에 따라 세 가지로 분류할 수 있다.
지도기반 (supervised) 머신 러닝은 객체의 특징을 기술하는 입력값 (input) 과 객체의 정답인 출력값 (output) 이 쌍으로 존재할 때 이를 이용하는 방식이다.
머신 러닝 모델은 입력값으로부터 출력값을 예측하기에 유용한 정보를 학습한다.
품사 판별의 경우, 단어 - 품사 쌍으로 이뤄진 학습 데이터를 이용하여, 단어열이 입력되었을 때 이에 해당하는 품사열을 판별하는 정보를 모델이 학습한다.
이후, 새로운 단어열이 입력되면 적절한 품사열을 출력한다.
비지도기반 (unsupervised) 머신 러닝은 데이터에 입력값만 존재할 때 쓰는 방법으로, 품사 판별의 경우, 두 단어가 앞 뒤에 등장하는 단어들의 분포가 비슷하면 하나의 품사로 이를 인식하는 Brown clustering (\citep{brown1992class}) 방법이 이에 해당한다.
강화학습 (reinforcement learning) 은 각 입력값에 대한 출력값은 주어지지 않았지만, 출력값에 대한 리워드 (reward) 를 정의할 수 있을 때 이용하는 방식이다.
출력값에 대하여 피드백을 줄 수 있는 도메인에서 이용할 수 있다.
대화 시스템에서 출력된 답변에 대한 피드백을 이용하여 답변 출력 방법을 수정하는 모델이 이에 해당한다 \citep{mo2018personalizing, singh2000reinforcement, li2016deep}. 

자연어처리의 많은 연구는 지도기반 머신러닝을 이용한다.
모델이 적용될 도메인의 입력값들을 포함하는 학습용 데이터를 구축한 뒤, 지도기반 모델을 이용하여 입력값과 출력값의 관계를 학습한다.
품사 판별의 경우, 단어 - 품사 쌍으로 이뤄진 말뭉치를 이용하여 한 단어의 앞, 뒤에 등장하는 문맥들을 고려하여 해당 단어의 품사를 추정하는 패턴을 학습한다.
품사 판별과 같은 과업은 텍스트 데이터를 벡터로 변환하는 전처리 과정에 이용되는 경우가 많기 때문에 결과값에 대한 피드백을 정의하기가 어렵기 때문에 지도기반 머신러닝 방법이 적합하다.

그러나 지도기반 머신러닝 방법을 이용하는 자연어처리 모델들은 공통적으로 다음과 같은 어려움이 있다.

\begin{enumerate}
    \item \textbf{Out of vocabulary problem} : 학습 데이터에 등장하지 않은 단어를 제대로 인식하지 못하는 문제이다.
    \item \textbf{Dearth of Data} : 과업에 적합한 학습 데이터를 마련하기 어렵다.
    \item \textbf{Noise} : 텍스트 데이터에는 노이즈가 존재한다.
\end{enumerate}

첫째, 미등록 단어 문제는 학습 데이터를 이용하는 지도기반 머신 러닝 방법에서는 필연적으로 발생하는 문제이다.
언어는 사용되는 시기와 도메인에 따라 다르기 때문에 한 종류의 학습 데이터에 모든 종류의 단어가 등장하지 않는다.
특히 품사 판별기와 같은 전처리 과업에서 자주 발생하는 문제로, 신조어와 같이 새롭게 만들어진 단어를 인식하지 못하여 문장이 잘못된 단어열로 분해된다.
이 결과를 이용하면 문장이나 문서를 잘못된 벡터로 표현되기 때문에 이후의 과업들의 성능이 저하된다.
라틴계 언어들은 단어를 띄어쓰기로 구분되며, 최근의 단어 임베딩 기법들을 이용하여 품사의 추정이 손쉽게 이뤄진다 (citep{turian2010word, mikolov2013efficient, collobert2011natural}).
하지만 단어의 경계가 띄어쓰기로 구분되지 않는 한국어, 중국어, 일본어에서는 미등록 단어를 제대로 인식하기 위해서는 추가적인 단어 사전 혹은 학습 데이터가 필요하다.

둘째, 학습 데이터가 존재하지 않거나, 이를 구축하기 어려운 경우들이 많다.
객체명 인식은 장소, 사람과 같이 단어의 의미적 클래스를 분류하는 과업이다.
지도학습 기반으로 객체명 인식 모델을 학습하려면 각 클래스가 태깅된 학습데이터가 필요하다.
하지만 객체명 인식이 필요한 도메인마다 단어 클래스가 다르기 때문에 새롭게 학습 데이터를 구축해야 한다.
영화 제목을 인식하는 객체명 인식 모델을 학습하기 위해서는 영화 제목이 태깅된 학습데이터가 필요하다.

셋째, 텍스트 데이터에는 노이즈가 존재한다.
특정 과업을 위해 태깅된 데이터를 구하기는 어렵지만, 레이블이 존재하지 않는 텍스트 데이터를 구하는 것은 어렵지 않다.
많은 양의 텍스트 데이터는 사람에 의하여 제작되기 떄문에 띄어쓰기 오류나 철자법 오류가 존재한다.
사전을 이용하는 품사 판별기는 철자법이 틀린 단어를 제대로 인식할 수 없다.
또한 띄어쓰기가 제대로 지켜지지 않으면 단어 간의 경계를 제대로 인식하지 못한다.
자연어처리 과업의 성능을 높이기 위해서는 노이즈를 제거하는 과정이 필요하다.

이 논문에서는 다음의 자연어처리 과업에서 발생하는 위의 세 가지 문제를 해결하는 방법들을 제안한다.

\begin{enumerate}
    \item Korean space error correction
    \item Enhancing part of speech tagging with unsupervised word extraction
    \item Keyword extraction
\end{enumerate}

첫째, 한국어 문서의 띄어쓰기 오류를 교정한다.
한국어는 띄어쓰기 오류가 일부 포함되어도 가독이 어렵지 않기 때문에 띄어쓰기 오류가 자주 발생한다.
띄어쓰기 오류 교정은 글자 단위의 순차적 판별 (sequential labeling) 문제이기 때문에 순차적 판별 알고리즘들이 이용될 수 있다.
이들은 띄어쓰기 오류가 없는 학습 데이터로부터 글자열에 가장 적합한 띄어쓰기 품사열을 출력한다.
그러나 도메인마다 사용되는 어휘가 다르기 때문에 각 도메인에 적합한 학습 데이터를 마련해야 한다.
3 장에서 일부 띄어쓰기 오류가 포함된 학습 데이터로도 안정적인 띄어쓰기 교정을 하는 방법을 제안한다.
또한 학습 데이터의 띄어쓰기 오류 수준에 따른 교정 능력의 변화도 확인한다.

둘째, 단어 추출 기법을 이용하여 품사 판별의 성능을 높인다.
품사 판별은 미등록단어 문제를 자주 겪는다.
특히 단어의 경계가 명확히 기록되지 않는 한국어에서는 미등록단어가 작은 단위의 형태소들로 분해되는 문제가 자주 발생한다.
이를 방지하기 위하여 한국어 품사 판별기들은 사용자가 직접 사전을 추가하는 기능을 제공하지만, 사용자 사전의 구축은 사용자의 몫이다.
단어 추출 기법은 문서 집합에서 통계 기반으로 단어를 추출하는 방법이다.
4 장에서 단어 추출 기법을 품사 판별 과업 모델에 추가하여 각 도메인에 적합한 사용자 사전을 스스로 구축하는 품사 판별기를 제안한다.

셋째, 비지도학습 기반 키워드 추출 기법을 제안한다.
키워드 추출 기법은 문서 요약에 이용되는 대표적인 방법이다.
하지만 각 문서나 문서 집합에 적합한 키워드가 무엇인지 태깅된 학습 데이터는 잘 존재하지 않기 때문에 비지도학습 기반의 방법들이 제안되었다.
키워드 추출 기법은 데이터의 성질에 따라 다르게 접근해야 한다.
문서 집합이 동일한 주제에 대한 문서들로 구성되었을 경우 (homogeneous data), TextRank (\citep{mihalcea2004textrank}) 같은 그래프 랭킹 기반 알고리즘이 이용될 수 있다.
하지만 TextRank 는 단어열이 제대로 분해되었을 경우 잘 작동한다.
5.1 장에서 데이터기반으로 미등록단어 문제를 해결하며 키워드를 추출하는 한국어 텍스트를 위한 그래프 갱킹 기반 방법을 제안한다.
문서 집합이 서로 다른 주제의 문서들로 구성된 경우 (heterogeneous data), 그래프 랭킹 기반 알고리즘은 빈도수가 높은 일반적인 단어들을 키워드로 선택한다.
이를 해결하기 위해서 문서 집합을 주제별로 구분하여야 한다.
5.2 장에서 문서 군집화 기법을 이용하여 문서 집합을 분류한 뒤, 각 문서 군집을 구분할 수 있는 단어를 키워드로 선택하는 방법을 제안한다.

이 논문에서 제안하는 방법들은 순차적 판별 알고리즘, 단어 임베딩 기법, 그리고 그래프 랭킹 알고리즘이 이용된다.
2 장에서는 이에 대하여 살펴본다.

%%%%%%%%%%%%%%%%%%%%%%%%%%%%%%%%%%%%%%%%%%%%%%%%%%%%%%%%%%%%%%%%%%%%%%%%%%%%%%%
\section{Related work}



%%%%%%%%%%%%%%%%%%%%%%%%%%%%%%%%%%%%%%%%%%%%%%%%%%%%%%%%%%%%%%%%%%%%%%%%%%%%%%%
\section{Prudent space correction and influence noise level of training data}




%%%%%%%%%%%%%%%%%%%%%%%%%%%%%%%%%%%%%%%%%%%%%%%%%%%%%%%%%%%%%%%%%%%%%%%%%%%%%%%
\section{Enhancing part of speech tagging with unsupervised word extraction}



%%%%%%%%%%%%%%%%%%%%%%%%%%%%%%%%%%%%%%%%%%%%%%%%%%%%%%%%%%%%%%%%%%%%%%%%%%%%%%%
\section{Keyword extraction in unsupervised manner}


%%%%%%%%%%%%%%%%%%%%%%%%%%%%%%%%%%%%%%%%%%%%%%%%%%%%%%%%%%%%%%%%%%%%%%%%%%%%%%%
\subsection{Graph ranking based method for homogeneous texts}



%%%%%%%%%%%%%%%%%%%%%%%%%%%%%%%%%%%%%%%%%%%%%%%%%%%%%%%%%%%%%%%%%%%%%%%%%%%%%%%
\subsection{Clustering and classification based method for heterogeneous texts}



%%%%%%%%%%%%%%%%%%%%%%%%%%%%%%%%%%%%%%%%%%%%%%%%%%%%%%%%%%%%%%%%%%%%%%%%%%%%%%%
\bibliography{reference}

\end{document}